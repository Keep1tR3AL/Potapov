\documentclass[11pt]{article}
\usepackage{amsmath,amssymb,amsthm}
\usepackage{algorithm}
\usepackage[noend]{algpseudocode} 

%---enable russian----

\usepackage[utf8]{inputenc}
\usepackage[russian]{babel}

% PROBABILITY SYMBOLS
\newcommand*\PROB\Pr 
\DeclareMathOperator*{\EXPECT}{\mathbb{E}}


% Sets, Rngs, ets 
\newcommand{\N}{{{\mathbb N}}}
\newcommand{\Z}{{{\mathbb Z}}}
\newcommand{\R}{{{\mathbb R}}}
\newcommand{\Zp}{\ints_p} % Integers modulo p
\newcommand{\Zq}{\ints_q} % Integers modulo q
\newcommand{\Zn}{\ints_N} % Integers modulo N

% Landau 
\newcommand{\bigO}{\mathcal{O}}
\newcommand*{\OLandau}{\bigO}
\newcommand*{\WLandau}{\Omega}
\newcommand*{\xOLandau}{\widetilde{\OLandau}}
\newcommand*{\xWLandau}{\widetilde{\WLandau}}
\newcommand*{\TLandau}{\Theta}
\newcommand*{\xTLandau}{\widetilde{\TLandau}}
\newcommand{\smallo}{o} %technically, an omicron
\newcommand{\softO}{\widetilde{\bigO}}
\newcommand{\wLandau}{\omega}
\newcommand{\negl}{\mathrm{negl}} 

% Misc
\newcommand{\eps}{\varepsilon}
\newcommand{\inprod}[1]{\left\langle #1 \right\rangle}

 
\newcommand{\handout}[5]{
  \noindent
  \begin{center}
  \framebox{
    \vbox{
      \hbox to 5.78in { {\bf Научно-исследовательская практика} \hfill #2 }
      \vspace{4mm}
      \hbox to 5.78in { {\Large \hfill #5  \hfill} }
      \vspace{2mm}
      \hbox to 5.78in { {\em #3 \hfill #4} }
    }
  }
  \end{center}
  \vspace*{4mm}
}

\newcommand{\lecture}[4]{\handout{#1}{#2}{#3}{Scribe: #4}{Вычисление символа Якоби #1}}

\newtheorem{theorem}{Теорема}
\newtheorem{lemma}{Лемма}
\newtheorem{definition}{Определение}
\newtheorem{corollary}{Следствие}
\newtheorem{fact}{Факт}

% 1-inch margins
\topmargin 0pt
\advance \topmargin by -\headheight
\advance \topmargin by -\headsep
\textheight 8.9in
\oddsidemargin 0pt
\evensidemargin \oddsidemargin
\marginparwidth 0.5in
\textwidth 6.5in

\parindent 0in
\parskip 1.5ex

\begin{document}

\lecture{}{Лето 2020}{}{Потапов Владислав}

\section{Теория}

Символ Якоби — теоретико-числовая функция двух аргументов, введённая К. Якоби в 1837 году. Является квадратичным характером в кольце вычетов.

\subsection{Определение}
Пусть $P$ — нечётное, большее единицы число и $P=p_1p_2...p_n$ — его разложение на простые множители (среди  $p_{1}...p_{n}$ могут быть равные). Тогда для произвольного целого числа $a$ символ Якоби определяется равенством:$(\frac{a}{P})=(\frac{a}{p_1})(\frac{a}{p_2})...(\frac{a}{p_n})$,где $(\frac{a}{p_i})$ — символы Лежандра.

По определению считаем, что $(\frac{a}{1})=1$ для всех $a$.
\subsection{Свойства}
\begin{itemize}
\item \textit{Мультипликативность} $(\frac{ab}{P})=(\frac{a}{P})(\frac{b}{P})$
	\begin{itemize}
	\item В частности, $(\frac{a^2b}{P})=(\frac{b}{P})$
	\end{itemize}
\item \textit{Периодичность}, если $a=b(mod P)$, то $(\frac{a}{P})=(\frac{b}{P})$
\item $(\frac{1}{P})=1$
\item $(\frac{-1}{P})=(-1)^\frac{P-1}{2}$
\item $(\frac{2}{P})=(-1)^\frac{P^2-1}{8}$
\item Если $Q$- нечетное натуральное число, взаимно простое с P, то $(\frac{Q}{P})(\frac{P}{Q})=(-1)^{(\frac{P-1}{2})(\frac{Q-1}{2})}$
\item Символ Якоби $(\frac{a}{P})$ равен знаку перестановки приведённой системы вычетов по модулю $P$, которая задаётся как умножение элементов этой группы на $a$(где $a$ обязательно взаимно просто с $P$).
\end{itemize}

\section{Алгоритм}
Символ Якоби практически никогда не вычисляют по определению. Чаще всего для вычисления используют свойства символа Якоби, главным образом — квадратичный закон взаимности.

\begin{algorithm}[ph]
	\caption{Алгоритм вычисления символа Якоби}
	\label{alg:AlgName}
	\textbf{Входные данные:} $a$ — целое число, $b$ — натуральное, нечётное, больше единицы\\
	\textbf{Выходные данные:} $(\frac{a}{b})$ — символ Якоби
	
	\begin{algorithmic}
		
		\State 1. \textit{(проверка взаимной простоты)}.\\
		\ \ \ Если НОД $(a,b)$ не равен единице, то\\
		\ \ \ \ \ \ \Return{0}
		\State 2. \textit{(инициализация)}. $r:=1$
		\State 3. \textit{(переход к положительным числам)}.\\
		\ \ \ Если $a<0$ то\\
		\ \ \ \ \ \ $a:=-a$\\
		\ \ \ \ \ \ Если $b (mod 4) = 3$ то\\
		\ \ \ \ \ \ \ \ \ $r:=-r$\\
		\ \ \ Конец если
		\State 4. \textit{(избавление от четности)}.$t:=0$\\
		\ \ \ Цикл пока $a$ - четное\\
		\ \ \ \ \ \ $t:=t+1$\\
		\ \ \ \ \ \ $a:=a/2$\\
		\ \ \ Конец цикла\\
		\ \ \ Если $t$ - нечетное, то\\
		\ \ \ \ \ \ Если $b(mod 8)=3$ или $5$, то\\
		\ \ \ \ \ \ \ \ \ $r:=-r$\\
		\ \ \ Конец если
		\State 5. \textit{(квадратичный закон взаимности)}.\\ 
		\ \ \ Если $a(mod 4) = b (mod 4) = 3$, то\\
		\ \ \ \ \ \ $r:=-r$.\\
		\ \ \ Конец если\\
  		\ \ \ $c:=a$; $a:=b (mod c)$; $b:=c$.
		\State 6. \textit{(выход из алгоритма?)}\\
		\ \ \ Если $a$ не равно $0$, то\\
		\ \ \ \ \ \ Идти на шаг 4\\
		\ \ \ Иначе\\
		\ \ \ \ \ \ \Return{$r$} 
	\end{algorithmic}

\end{algorithm}
В алгоритме везде берётся наименьший положительный вычет (то есть остаток от деления).
На четвёртом шаге используется мультипликативность символа Якоби, а затем вычисляется символ Якоби $(\frac{2}{b})=(-1)^\frac{b^2-1}{8}$. Чтобы избежать лишнего возведения в степень, проверяется только остаток от деления $b$ на 8. На пятом шаге тоже вместо возведения в степень используется проверка остатков от деления. Сложность алгоритма равна $O(log(a)*log(b))$ битовых операций.

\section{Машина}
Компьютер, на котором проходили тестирования, имеет следующие основные характеристики:
\begin{itemize}
\item Процессор: \textbf{Intel(R) Core(TM) i5-9600KF CPU @ 3.7 GHz (6 ядер и 6 потоков)}
\item Видеоадаптер: \textbf{NVIDIA GeForce RTX 2060 SUPER}
\item Оперативная память: \textbf{16GB}
\end{itemize}
\section{Анализ}

\begin{tabular}{|c|c|c|c|}
\hline
a & b & Время выполнения алгоритма Sage & Время выполнения моего алгоритма\\
\hline
$17^{5397}$ & $113537^{753}$ & 0.00s & 0.05s\\
$137^{5397}$ & $113537^{7531}$ & 0.00s & 0.27s\\
$1957^{5398}$ & $12345^{7953}$ & 0.01s & 0.56s\\
$9997^{9998}$ & $79945^{99999}$ & 0.03s & 2.09s\\
$99987^{59998}$ & $799495^{999999}$ & 0.41s & 109.27s\\
\hline
\end{tabular}

Тесты я проводил в онлайн компеляторе CoCalc. Исходя из результатов тестирования можно сделать вывод, что алгоритм sage при вычислнии символа Якоби затрачивает намного меньше времени, чем мой алгоритм.

\paragraph{Bibliography.}
\begin{itemize}
\item Львовский С.М. - Набор и верстка в системе LATEX. 5-е изд. 2014.
\item Wikipedia: Wikipedia, The Free Encyclopedia [online], http://www.wikipedia.org
\item Zimmermann P., Casamayou A., Cohen N. et al. - Computational Mathematics with SageMath. 2018
\end{itemize}

\end{document}